\documentclass[10pt, a4paper]{article}
\usepackage[paper=a4paper, left=1.5cm, right=1.5cm, bottom=1.5cm, top=3cm]{geometry}
\usepackage[utf8]{inputenc}
\usepackage[T1]{fontenc}
\usepackage[spanish]{babel}
\usepackage{indentfirst}
\usepackage{fancyhdr}
\usepackage{lastpage}
\usepackage{calc}
\usepackage{caratula}
\usepackage{marvosym} % para \Faxmachine !
\usepackage{graphicx}
\usepackage{float}
\usepackage{algpseudocode}
\usepackage{multicol}
\usepackage[hidelinks]{hyperref}
\graphicspath{{imagenes/}}
%\sloppy
\parskip=5pt % 10pt es el tamano de fuente

\usepackage{stringenc}
\usepackage{pdfescape}
\usepackage{listings}
\lstset{
	extendedchars=true,
    literate={á}{{\'a}}1 {é}{{\'e}}1 {í}{{\'i}}1 {ó}{{\'o}}1 {ú}{{\'u}}1 {ñ}{{\~n}}1,
    basicstyle=\ttfamily\small,
    tabsize=2,
    breaklines=true,
    breakatwhitespace=true,
}


\begin{document}
\title{PLP - TP1}
\materia{Paradigmas de Lenguajes de Programación}
\submateria{Primer cuatrimestre 2017}
\titulo{}
\subtitulo{TP1}
\grupo{Grupo: Pescado Ravioloso}
\integrante{Alejandro González}{...}{gonzalezalejandro1592@gmail.com}
\integrante{Diego Sueiro}{75/90}{dsueiro@gmail.com}
\integrante{Malena Ivnisky}{421/12}{malenaivnisky@gmail.com}

\maketitle

\newpage

\subsection{Código}
\lstinputlisting[language=Haskell, firstline=24, lastline=122]{../NavesEspaciales.hs}

\subsection{Tests}
\lstinputlisting[language=Haskell, firstline=53, lastline=124]{../Main.hs}

\end{document}
